% !TEX root = memoria.tex

\documentclass[12pt]{article}

%%% Imports %%%
\usepackage[normalem]{ulem} %% Subrayado multilinea
\usepackage{subfiles} %% Proyecto multi archivo
\usepackage[spanish,es-tabla]{babel} %% Idioma español
\usepackage[utf8]{inputenc} %% Permitir entrada utf-8
\usepackage{marvosym} %% Símbolo €
\usepackage[margin=2cm]{geometry} %% Márgenes
\usepackage{fancyhdr} %% Encabezados y pies de página
\usepackage{titling} %% Portada
\usepackage{csquotes} %% Requerido por biblatex
\usepackage[sorting=none]{biblatex} %% Bibliografía
\usepackage{graphicx} %%Insertar imagenes
\usepackage{float} %% Floats
\usepackage{microtype} %% Justificar cuando hay carácteres especiales
\usepackage[export]{adjustbox}
\usepackage{varwidth}



%%% Configuración del preamble %%%

%%% Configuración global %%%
\graphicspath{ {./imagenes/} }
\addbibresource{references.bib} %% Importar bibliografía

%% Tabla de contenidos
\addto\captionsspanish{
    \renewcommand{\contentsname}
    {Índice de Contenidos}
}

%% Icono de los bullets en las listas
\renewcommand{\labelitemi}{$\bullet$}
\renewcommand{\labelitemii}{$\circ$}

%% Formato en referencias de la fecha de visita de la web
\DeclareSourcemap{
    \maps{
        \map[overwrite]{
            \step[fieldsource=urldate,
                match=\regexp{([0-9]{2})\-([0-9]{2})\-([0-9]{4})},
                replace={$3-$2-$1}, final]
        }
    }
}

\emergencystretch=2em


%%% Documento %%%
\begin{document}

%%% Portada %%%
\subfile{sections/0-prev/010-portada.tex}
\thispagestyle{empty}
\newpage

%%% Resumen y Abstract %%%
\section*{Resumen y Abstract}
\subfile{sections/0-prev/020-resumen-y-abstract.tex}
\newpage

%%% Índice de Contenidos %%%
\tableofcontents{}
\newpage

%%% Lista de Acrónimos %%%
\section*{Lista de Acrónimos}
\subfile{sections/0-prev/030-lista-de-acronimos.tex}
\newpage

%%% Introducción %%%
\section{Introducción}
\subfile{sections/1-introduccion/100-introduccion.tex}
\newpage

%%% Antecedentes o Marco Tecnológico %%%
\section{Antecedentes o Marco Tecnológico}
\subfile{sections/2-antecedentes-marco-tecnologico/200-antecedentes-marco-tecnologico.tex}

\subsection{Agricutura de Precisión}
\subfile{sections/2-antecedentes-marco-tecnologico/210-agricultura-de-precision.tex}

\subsection{Internet de las Cosas (IoT)}
\subfile{sections/2-antecedentes-marco-tecnologico/220-internet-de-las-cosas.tex}

\subsection{Computación en la Nube}
\subfile{sections/2-antecedentes-marco-tecnologico/230-computacion-nube.tex}

\subsection{Hardware IoT}
\subfile{sections/2-antecedentes-marco-tecnologico/240-hardware-iot.tex}

\subsection{Protocolos de Comunicación para IoT}
\subfile{sections/2-antecedentes-marco-tecnologico/250-protocolos-topologias.tex}

\subsection{WIP: Ejemplos de Agricultura de Precisión}
\subfile{sections/2-antecedentes-marco-tecnologico/260-ejemplos-agricultura.tex}

%%% Especificaciones y restricciones de diseño %%%
\section{Especificaciones y restricciones de diseño}
\subfile{sections/3-especificaciones-restricciones/300-especificaciones-restricciones.tex}

\subsection{Descripción de las tecnologías y servicios necesarios}
\subfile{sections/3-especificaciones-restricciones/310-descripcion-tecnologias.tex}

\subsubsection{Amazon Web Services (AWS)}
\subfile{sections/3-especificaciones-restricciones/311-aws.tex}

\subsubsection{Terraform}
\subfile{sections/3-especificaciones-restricciones/312-terraform.tex}

\subsubsection{Python y AWS SDK para Python (boto3)}
\subfile{sections/3-especificaciones-restricciones/313-python-aws-sdk.tex}

%%% Descripción de la solución propuesta %%%
\section{Descripción de la solución propuesta}
\subsection{Solución propuesta}
\subfile{sections/4-solucion-propuesta/410-solucion-propuesta.tex}


\subsection{Desarrollo de la solución}
\subsubsection{Configuración inicial. Usuario terraform y credenciales de AWS}
\subfile{sections/4-solucion-propuesta/421-conf-inicial.tex}

\subsubsection{Código de la infraestructura}
\subfile{sections/4-solucion-propuesta/422-codigo-infra.tex}

\subsubsection{Código del simulador}
\subfile{sections/4-solucion-propuesta/423-codigo-simulador.tex}

\subsubsection{Código auxiliar}
\subfile{sections/4-solucion-propuesta/424-codigo-aux.tex}

\subsubsection{Modelado de la base de datos (Elasticsearch)}
\subfile{sections/4-solucion-propuesta/425-modelado-elasticsearch.tex}

\subsection{Despliegue de la solución}
\subfile{sections/4-solucion-propuesta/430-despliegue.tex}

%%% Resultados %%%
\section{Pruebas realizadas}
\subfile{sections/5-resultados/510-pruebas.tex}
\subsection{Tiempo de despliegue y destrucción de la infraestructura}
\subfile{sections/5-resultados/511-prueba-despliegue.tex}
\subsection{Comunicación hacia los dispositivos}
\subfile{sections/5-resultados/512-prueba-comunicacion.tex}
%% HACER UN PYTHON DE EJEMPLO QUE SE SUSCRIBA A UN TOPIC, ESCRIBIRLE DESDE AWS PARA QUE SE VEA CONECTIVIDAD DE AWS AL DISPOSITIVO
\subsection{Representación de los datos}
\subfile{sections/5-resultados/513-prueba-representacion.tex}
%% AQUI METER CAPTURITAS DEL GRAFANA Y EXPLICAR PORQUE USAR GRAFANA PARA MOSTRAR RESULTADOS.


%%% Planos %%%
% \section{Diagramas de arquitectura}

%%% Presupuesto %%%
\section{Presupuesto}
\subfile{sections/6-presupuesto/610-presupuesto.tex}

%%% Conclusiones y trabajos futuros%%%
\section{Conclusiones y trabajos futuros}
\subfile{sections/7-conclusiones-trabajos-futuros/710-conclusiones-trabajos-futuros.tex}

%%% Referencias %%%
\newpage
\setcounter{secnumdepth}{0}
\section{Referencias}
\printbibliography[heading=none]

\end{document}
