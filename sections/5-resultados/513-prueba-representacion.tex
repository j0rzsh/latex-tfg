\documentclass[../../memoria.tex]{subfiles}

\begin{document}

\paragraph{}
La última prueba que se ha realizado sobre el sistema incluye una herramienta externa de visualización de datos: Grafana \cite{grafana}. Grafana es una herramienta que puede integrarse con una gran cantidad de fuentes de datos como Elasticsearch. Posee también un motor de alertas que permitiría ejecutar reglas o realizar acciones basándose en las métricas recolectadas. Es una herramienta gratuita y de código abierto.

\paragraph{}
Grafana se ejecutará en local, fuera del entorno del \textit{MVP} por dos motivos: no perder la configuración cuando el entorno sea destruido y tener una herramienta conectada al sistema, para demostrar la compatibilidad del mismo.

\paragraph{}
Para el \textit{MVP}, se van a generar cuatro cuadros de mando (\textit{dashboards}) diferentes, uno para cada sensor y cada uno de ellos con diferentes visualizaciones de las métricas del mismo.Por simplificar, sólo se va a generar de manera completa el \textit{dashboard} perteneciente al sensor1. El resto se prepararán pero no se completará su implementación en esta prueba. Esta herramienta sería una de las que, en un escenario real, sería operada por el encargado del campo de cultivo. Es por tanto una herramienta cuya configuración debe orientarse al usuario y debe ser fácil de utilizar.

\paragraph{}
A continuación, se incluye una captura de la interfaz de Grafana, configurada para recibir datos de Elasticsearch y representarlos en los diferentes \textit{dashboards}. En el apartado de trabajos futuros, se detallarán posibles desarrollos a realizar utilizando esta herramienta.

\begin{figure}[H]
    \centering
    \includegraphics[width=1\columnwidth]{pruebaRepresentacion1.png}
    \caption{Interfaz de Grafana mostrando las métricas del sensor1}
    \label{fig:pruebaRepresentacion1}
\end{figure}

\paragraph{}
La figura anterior no es más que un ejemplo de representación de las métricas de uno de los sensores IoT. Se ha configurado para mostrar los datos en tiempo real de las métricas que reportan los dispositivos hacia la nube. Se trata de un pequeño ejemplo de qué hacer con los datos una vez indexados en Elasticsearch.
\end{document}
