\documentclass[../../memoria.tex]{subfiles}

\begin{document}


\paragraph{}
Como se ha mencionado anteriormente en el capítulo 4.2 Desarrollo de la Solución, el presupuesto para este proyecto es de 0€ al poder utilizar la capa gratiuita de AWS \cite{awsfreetier}. Es interesante desglosar qué recursos se están utilizando para el \textit{MVP} y cuánto costarían mensualmente dichos recursos fuera de la capa gratiuita de AWS.

\paragraph{}
Los recursos más significativos son:

\begin{enumerate}

    \item Para el sistema (excluidos recursos como por ejemplo IAM, ya que los que se crean son gratuitos siempre):

          \begin{itemize}
              \item Elasticsearch Service:
                    \begin{itemize}
                        \item t2.small.elasticsearch instance.
                        \item EBS Storage (10GB)
                    \end{itemize}

              \item AWS Lambda Function
              \item AWS Lambda Layer
              \item 4 IoT Things
              \item 4 IoT Topic Rules
          \end{itemize}

    \item Para el simulador (excluidos recursos como por ejemplo IAM o tablas de ruta, ya que los que se crean son gratuitos siempre):
          \begin{itemize}
              \item EC2 Instance t2.micro
              \item AWS VPC
              \item AWS Subnet
              \item AWS Elastic IP
              \item AWS Internet Gateway
          \end{itemize}
\end{enumerate}

\paragraph{}
El precio de dichos recursos fuera de la capa gratuita es (para la región de Irlanda eu-west-1):

\begin{enumerate}

    \item Para el sistema:

          \begin{itemize}
              \item Elasticsearch Service \cite{awselasticpricing}:
                    \begin{itemize}
                        \item t2.small.elasticsearch instance: \$0.039/hora
                        \item EBS Storage (10GB): \$0.149 GB/month
                    \end{itemize}

              \item AWS Lambda \cite{awslambdapricing}
                    \begin{itemize}
                        \item \textit{Provisioned Concurrency}: \$0.000004646 por cada GB-segundo.
                        \item Peticiones \$0.20 por 1M de peticiones
                        \item Duration \$0.0000108407 por cada GB-segundo (depende de la cantidad de memoria reservada para la ejecución de la función)
                    \end{itemize}
              \item AWS IoT \cite{awsiotpricing}
                    \begin{itemize}
                        \item Conectividad: \$0.08 por cada millón de minutos conectado. (Es necesario enviar un mensaje \textit{keep-alive} desde cada 20 minutos hasta cada 30 segundos para mantener conectividad, pero no es necesario ya que los mensajes de los dispositivos en el \textit{MVP} envían desde cada 5 hasta cada 10 segundos)
                        \item Mensajes: hasta 1000 millones de mensajes, \$1/millón de mensajes
                        \item Reglas activadas: \$0.15 por cada millón de reglas activadas. En este caso, cada mensaje enviado activa una regla.
                    \end{itemize}
          \end{itemize}

    \item Para el simulador:
          \begin{itemize}
              \item AWS EC2 \cite{awsec2pricing}
                    \begin{itemize}
                        \item Instancia t2.micro (On-Demand): \$0.0126/hora
                    \end{itemize}
              \item AWS VPC, AWS Subnet, AWS Elastic IP, AWS Internet Gateway \cite{awsvpcpricing}
                    \begin{itemize}
                        \item Todos estos recursos son gratuitos salvo la Elastic IP que únicamente se cobra el tiempo que no está asignada a una instancia. Para este caso, siempre que exista va a estar asignada a la instancia. Por lo tanto, \$0.
                    \end{itemize}
          \end{itemize}
\end{enumerate}

\paragraph{}
Teniendo en cuenta todos estos datos, y considerando:
\begin{itemize}
    \item Un mes de 31 días. Por lo tanto, 744 horas.
    \item El sistema estará funcionando 24/7.
    \item Se crean 4 dispositivos que envían mensajes cada 5 segundos (frecuencia máxima de envío). Esto es 535680 mensajes al mes por cada dispositivo.
    \item Cada mensaje enviado, dispara una regla de AWS IoT. Esto es 535680 reglas disparadas al mes por cada dispositivo.
    \item Cada Lambda se va a ejecutar durante 0.5 segundos (se ejecuta una por cada regla disparada de AWS IoT) y se reserva 1GB de memoria por cada ejecución. Esto es 535680 ejecuciones al mes por cada dispositivo, lo que implica 267840 GB-segundos al mes.
\end{itemize}
El precio del sistema desplegado para el \textit{MVP} fuera de la capa gratuita de AWS sería:

\begin{itemize}
    \item Para el sistema:
          \begin{itemize}
              \item Elasticsearch: \$0.039 * 744 horas + \$0.149 * 10 GB-mes = \$30.51 al mes.
              \item AWS Lambda:
                    \begin{itemize}
                        \item \textit{Provisioned Capacity} = \$0.000004646 * 267840 GB-segundos = \$1.24 al mes
                        \item Peticiones = \$0.20 * 535680/1M = \$0.107136 al mes
                        \item Duración = \$0.0000108407 * 267840 GB-segundos = \$2.9036 al mes
                        \item Total: 4 dispositivos * (\$1.24 + \$0.107136 + \$2.9036) = \$17 al mes
                    \end{itemize}
              \item AWS IoT:
                    \begin{itemize}
                        \item Conectividad: \$0.08 * 44640/1M = \$0.035712 al mes
                        \item Mensajes: \$1 *  535680/1M = \$0.53568 al mes
                        \item Reglas: \$0.15 * 535680/1M = \$0.080352 al mes
                        \item Total: 4 dispositivos * (\$0.035712 + \$0.53568 + \$0.080352) = \$2.61 al mes
                    \end{itemize}
              \item Total: \$30.51 + \$17 + \$2.61 = \$50.12
          \end{itemize}
    \item Para el simulador:
          \begin{itemize}
              \item AWS EC2: \$0.0126 * 744 horas = \$9.37 al mes
              \item Total: \$9.37 al mes
          \end{itemize}
\end{itemize}

Por lo tanto, sumando los costes del sistema y del simulador, el gasto total al mes sería de aproximadamente \$50.12 + \$9.37 = \$59.49 al mes.

\paragraph{}
Sin embargo, este gasto no es real, ya que en un sistema real, los dispositivos no enviarían cada 5 segundos, sino cada 5 minutos, a modo de ejemplo. Por contra, no se han tenido en cuenta los posibles gastos que se generarían por el tráfico saliente de AWS, aunque sería prácticamente despreciable.

\paragraph{}
La conclusión que se puede obtener de este presupuesto de ejemplo es que se puede conseguir un sistema muy complejo y elaborado con un presupuesto muy ajustado. Para ello es necesario conocer cómo se tarifa en la nube pública (AWS en este caso), y qué servicios son interesantes para utilizar teniendo en cuenta estas tarificaciones. Para trabajos futuros, si este sistema se despliega en un entorno de producción, sería necesario establecer otros procedimientos como backups, apagado y encendido de recursos, reserva de instancias, etc.
\end{document}
