\documentclass[../../memoria.tex]{subfiles}

\begin{document}

\paragraph{}
Como se ha mencionado al inicio de este capítulo, para el resto de herramientas o aplicaciones necesarias para la realización del proyecto, se empleará el lenguaje Python. Elegir este lenguaje está motivado por la librería existente para trabajar con AWS: boto3 \cite{boto3}. Esta librería representa el SDK para trabajar con AWS para Python. Es posible utilizar otros lenguajes como NodeJS o Java, pero normalmente resulta más sencillo utilizar Python.

\paragraph{}
Mediante el uso de los servicios y las herramientas aquí mencionados, se tratará de ofrecer diferentes respuestas a un sistema IoT sobre AWS. La idea principal es crear un sistema base que posea todos los elementos necesarios para un funcionamiento mínimo y ofrecer alternativas añadiendo o cambiando dichos elementos para crear sistemas más complejos que se adecúen mejor a diferentes casos de usos más definidos.
\end{document}
