\documentclass[../../memoria.tex]{subfiles}

\begin{document}

\paragraph{}
Para la realización de este proyecto se persigue dar respuesta a los siguientes requisitos generales:

\begin{enumerate}
    \item Económicamente viable: se propondrá una solución que sea económicamente viable, cuyo presupuesto y gasto estimado se adapten a la magnitud del sistema a desplegar para cada caso de uso. Es por ello por lo que el sistema deberá de poder ajustarse en presupuesto a cualquier rango de gasto establecido, cumpliendo una serie de requisitos de rendimiento mínimos. Es por ello que se realizará un pequeño análisis del presupuesto en uno de los apartados de este proyecto.

    \item Completamente escalable: derivado del punto anterior, el sistema debe de ser definido de tal manera que escale de manera automática, o al menos, que permita este escalado, para cualquier tipo de caso de uso independientemente de la magnitud del sistema y la cantidad de datos enviados y/o recibidos y procesados. De esta forma el sistema podrá ser utilizado tanto con fuentes de datos pequeñas, como por ejemplo una única tierra de cultivo, como con grandes cantidades de nodos de envío de datos, como podría ser una empresa de producción y cultivo de tierras en masa.

    \item Adaptable y compatible a todo tipo de casos de uso similares: en este proyecto se describirá un caso de uso muy bien definido y cerrado, lo que no tiene que significar que la solución o el sistema definido no sea capaz de adaptarse a otros casos de uso similares pero diferentes de base. Se propondrá un sistema que sea capaz de adaptarse a cualquier situación de monitorización basada en IoT similar al caos de uso de este proyecto. Para ello se tratará de implementar el sistema de la manera más compatible posible con cualquier forma de enviar los datos a la nube. Se perseguirá el uso de \textit{endpoints} a modo de API que permitan la comunicación desde cualquier punto con acceso a Internet. De esta forma, será posible la integración con cualquier fabricante de hardware y no se cerrará la solución a un único tipo de dispositivos o fabricantes.

    \item Reutilizable: utilizar infraestructura como código hará que el sistema no solo sea más sencillo de actualizar y mantener, sino que será completamente reutilizable. Esto permitirá poder desarrollar el sistema una vez, pero desplegarlo las veces que se requiera, lo que supondrá un ahorro de costes una vez que el sistema esté definido. Además, la infraestructura como código permite versionarla y actualizarla de una manera muy simple y rápida sin ningún coste adicional al de los recursos de computación en \textit{Cloud} que se utilicen. Adicionalmente, cualquier desarrollo que se realice para este proyecto será distribuido de manera libre utilizando una licencia \textit{Open Source} \cite{repogithub}.

    \item Producto final de fácil utilización por todo tipo de perfiles: para que el sistema sea aceptado por parte de los clientes finales potenciales, deberá ser intuitivo y fácil de utilizar, que no requiera de mantenimiento por parte de dicho cliente final. Por su naturaleza, el sistema será utilizado por personas encargadas de labrar y mantener tierras de cultivo. El sistema tiene que adaptarse a la utilización por parte de cualquier tipo perfil, sea o no técnico, y no debe suponer un tiempo extra de mantenimiento para los clientes del producto. El producto final deberá de adaptarse a los clientes y no al contrario. Ya que este proyecto se enfoca en la infraestructura base, no se realizarán desarrollos de ningún tipo de interfaz que dé respuesta a esta restricción y/o especificación. Este desarrollo se marcará como trabajos futuros al de este proyecto.

\end{enumerate}

\paragraph{}
Aparte de las restricciones anteriormente mencionadas, también se tendrá en cuenta que:

\begin{enumerate}
    \item Para el desarrollo de este proyecto y la implementación del \textit{MVP} indicado en el capítulo de introducción, se pretende mantener a coste 0 la infraestructura que se despliegue para dar soporte al sistema.

    \item Una de las intenciones principales es mantener el coste lo más bajo posible incluso para un sistema en producción real. Para ello, como se ha mencionado anteriormente, se realizará un estudio sobre el presupuesto de la infraestructura propuesta para el sistema.
\end{enumerate}

\paragraph{}
Por ello, el \textit{MVP} implementado tendrá una funcionalidad mínima pero nunca podrá ser utilizado en un sistema en producción conservando las mismas especificaciones técnicas y de rendimiento, ya que será implementado para realizar una pequeña prueba de concepto que muestre el funcionamiento de la infraestructura con una cantidad de datos transferidos ínfima que nunca podrá dar solución a una problemática real de una tierra de cultivo. Sin embargo, sí se detallarán diferentes especificaciones técnicas de la infraestructura que deberán de modificarse para que el sistema pueda adaptarse correctamente a distintas necesidades de entornos productivos.

\paragraph{}
Por todo esto, se han estudiado las tres nubes públicas principales que existen actualmente: AWS, Azure y GCP. Tras una breve investigación sobre ellas, se ha llegado a la conclusión de utilizar AWS debido a que la capa gratuita de 1 año que ofrece posee los elementos clave que se necesitan para este sistema. Además, al ser la nube con mayor uso actualmente, posee también una comunidad de usuarios mayor que ayudará a la realización de este proyecto. Sin embargo, el hecho de implementar el sistema utilizando infraestructura como código junto con una herramienta que se agnóstica del proveedor \textit{Cloud} a utilizar permitirá migrar de una nube a otra de manera más sencilla, aunque no de manera trivial. Por el contrario, al elegir una única nube se va a generar un \textit{vendor lock-in} con dicha nube. Es por ello por lo que, como trabajos futuros a este mismo proyecto, se propondrá el estudio del mismo sistema, pero basado en las otras dos nubes mencionadas, Microsoft Azure y GCP.

\end{document}
