\documentclass[../../memoria.tex]{subfiles}

\begin{document}
\paragraph{}
En este trabajo de investigación, se va a tratar de proponer una solución al siguiente problema:

\paragraph{}
\textit{
    Necesidad de un sistema económicamente viable para llevar a cabo las diferentes tareas requeridas para una Agricultura de Precisión eficiente tanto en recursos como en presupuesto y que a su vez permita la utilización de manera sencilla por los operarios, los cuales no necesariamente dispondrán de un perfil técnico que conozca las diferentes tecnologías empleadas para el sistema.
}

\paragraph{}
Para dar una solución a esta problemática se va a tratar de integrar el paradigma IoT junto con la Computación en Nube o \textit{Cloud Computing}. También se van a estudiar posibles soluciones para la parte Hardware de la solución, aunque no es el objetivo principal de este trabajo.

\paragraph{}
Como objetivo o meta principal se va a tratar de dar una solución basada en Computación en Nube que encaje con dicho paradigma IoT. Se diseñará una infraestructura base que soporte el sistema, así como una arquitectura software que sirva para procesar y almacenar los datos recogidos de los dispositivos IoT.

\paragraph{}
Antes de poder presentar una solución eficiente que encaje en el momento actual en el que se encuentra la tecnología, es necesario realizar un extenso estudio del “Estado del Arte” actual para las tecnologías o paradigmas que se van a utilizar en dicha solución. A continuación, en el siguiente capítulo, se presentará este estudio que servirá de base para el diseño de la solución.

\paragraph{}
Es importante recalcar que el objetivo de este trabajo reside en el diseño y la implementación de una solución básica, que represente un \textit{Minimum Viable Product (MVP)} o Producto Viable Mínimo para la parte del sistema encargada de recoger e indexar los datos de los dispositivos IoT. Es por ello que al final del trabajo se expondrán posibles líneas de investigación o trabajos futuros, así como una descripción general de lo que debería ser un sistema completo y final.

\end{document}
