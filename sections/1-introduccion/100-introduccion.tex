\documentclass[../../memoria.tex]{subfiles}

\begin{document}
\paragraph{}
En este trabajo de investigación, se va a tratar de proponer una solución al siguiente problema:

\paragraph{}
\textit{
    Necesidad de un sistema económicamente viable para llevar a cabo las diferentes tareas requeridas para una agricultura de precisión eficiente tanto en recursos como en presupuesto y que a su vez permita su utilización de manera sencilla por los operarios, los cuales no necesariamente dispondrán de un perfil técnico que conozca las diferentes tecnologías empleadas en el sistema.
}

\paragraph{}
Para dar una solución a esta problemática se va a tratar de integrar el paradigma \textit{IoT} junto con la computación en la nube o \textit{Cloud Computing} en el dominio de aplicación de la agricultura. También se van a estudiar posibles soluciones para los elementos hardware de la solución, aunque no es el objetivo principal de este trabajo.

\paragraph{}
Como objetivo o meta principal se va a tratar de dar una solución basada en computación en la nube que encaje con el paradigma IoT. Se diseñará una infraestructura base que soporte el sistema, así como una arquitectura software que sirva para procesar y almacenar los datos recogidos de los dispositivos IoT.

\paragraph{}
Antes de poder presentar una solución que dé respuesta a la necesidad planteada, es necesario realizar un estudio sobre la actualidad de las tecnologias que se pretenden usar. Es decir, estudiar el estado del arte e investigar sobre ejemplos reales que implementen dichas tecnologías. Este paso será presentado en el siguiente apartado de este proyecto.

\paragraph{}
Es importante recalcar que el objetivo de este trabajo reside en el diseño y la implementación de una solución básica, que represente un \textit{Minimum Viable Product (MVP)} o producto viable mínimo para la parte del sistema encargada de recoger e indexar los datos de los dispositivos IoT. Es por ello que al final del trabajo se expondrán posibles líneas de investigación o trabajos futuros, así como una descripción general de lo que debería ser un sistema completo y final.

\paragraph{}
A continuación, en el capítulo 2. Antecedentes y Marco Tecnológico se va a proceder a explicar todo el contexto en el cual se basará este trabajo de investigacióny desarrollo. Se tratarán todos los aspectos tecnológicos relacionados y se describirán algunos ejemplos que soporten la idea que se pretende conseguir con el \textit{MVP} mencionado.

\paragraph{}
Tras analizar el marco tecnológico, se detallarán en el capítulo 3. Especificaciones y restricciones de diseño las tecnologías que se van a utilizar en el proyecto, otorgando un razonamiento a cada una de las elecciones que se harán para el desarrollo.

\paragraph{}
Una vez descritas las tecnologías a utilizar, se explicará la solución propuesta en el capítulo 4. Descripción de la solución propuesta. En ese capítulo se mostrará con todo detalle el diseño, desarrollo y despliegue de la solución.

\paragraph{}
Tras el despliegue de la solución propuesta, se realizarán sobre el sistema diferentes pruebas documentadas en el capítulo 5. Pruebas realizadas, que demuestren que el \textit{MVP} funciona como se espera. Tras estas pruebas, se detallará en el capítulo 6. Presupuesto un desglose del presupuesto de la infraestructura desplegada.

\paragraph{}
Finalmente en el capítulo 7. Conclusiones y trabajos futuros, se analizará en qué lugar se encuentra la idea inicial y se sentarán diferentes bases para establecer una línea de continuidad sobre la misma.

\end{document}
