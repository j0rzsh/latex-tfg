\documentclass[../../memoria.tex]{subfiles}

\begin{document}

\paragraph{}
Por último y para finalizar este capítulo, se van a mostrar ejemplos reales de agricultura de precisión que muestran casos de éxito de la utilización de la tecnología IoT en la agricultura.
(Farmobile?)

\paragraph{}
El primero de los ejemplos viene de la mano de Agrotech y utiliza los dispositivos Libelium \cite{libeliumusecase1}. En este caso de uso se ha desplegado un sistema de monitorización sobre un viñedo basado en IoT haciendo uso de dispositivos Libelium y la nube pública de Microsoft: Azure. Tiene como fin proveer de gran cantidad de datos para su posterior análisis y ayudar al sector de la agricultura en su digitalización.

\paragraph{}
Otro ejemplo es el documentado en \cite{inproceedings}. En este artículo, se ha implementado una pequeña granja conectada y se ha desarrollado una aplicación móvil que consulta los datos accediendo a una API REST. Se trata de un escenario real pero que incluye todo lo necesario para realizar una monitorización de una tierra de cultivo.

\paragraph{}
El último ejemplo que se presenta es uno de los mencionados en \cite{vodafones} y que utiliza la solución Sensing4Farming de Vodafone. En este caso se ha implantado un sistema de monitorización basado en en sensores que utilizan \textit{NarrowBand IoT} \cite{narrowband}. Se ha utilizado para un viñedo mediante una alianza con Emilio Moro \cite{emiliomoro}

\paragraph{}
Como se ha podido comprobar, hay multitud de formas de implementar un sistema de monitorización para tierras de cultivo utilizando IoT. La mayoría de los sistemas se centran en medidas como la temperatura, la humedad del terreno, la presión atmosférica, etc. La recopilación de estas medidas en diferentes partes de la tierra de cultivo permitiría una actuación diferenciada por cada una de estas partes, dependiendo de la necesidad de las mismas en cuanto a riego, tratamiento de las plantas, etc.

\end{document}
