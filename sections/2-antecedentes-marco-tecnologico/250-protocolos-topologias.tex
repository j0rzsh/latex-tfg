\documentclass[../../memoria.tex]{subfiles}

\begin{document}

\paragraph{}
De nuevo es necesario mencionar que, aunque este proyecto no esté enfocado a la implementación del Hardware IoT que realiza las mediciones y actuaciones, hay que tener en cuenta que el proyecto pueda ser implementado de manera completa en un entorno real. Es por ello por lo que en este apartado se van a tratar diferentes protocolos de comunicación enfocados a IoT y al bajo consumo que puedes ser de aplicación para este proyecto.

\paragraph{}
Asimismo, es interesante mencionar diferentes topologías de red que se pueden usar en IoT. Dependiendo de cada caso de uso específico en el que apliquen restricciones del terreno y el espacio, será necesario o más interesante emplear una u otra topología de red o modelo de comunicación.

\paragraph{}
Se van a describir los siguientes protocolos de comunicación: LoRa, WiMax, Zigbee, ¿6LowPAN?, HTTP, MQTT. También se van a describir las siguientes topologías de red y/o paradigmas de comunicación: topología en estrella, topología en malla y p2p.


\begin{itemize}
    \item \uline{LoRa}: es la capa física \cite{lora} \cite{loratech}

    \item \uline{WiMax}:

    \item \uline{Zigbee}:

    \item \uline{6LowPan}:

    \item \uline{Peer-to-peer}:

    \item \uline{HTTP}:

    \item \uline{MQTT}:

\end{itemize}

Topología en malla:

Topología en estrella:

\end{document}
