\documentclass[../../memoria.tex]{subfiles}

\begin{document}

\paragraph{}
De nuevo es necesario mencionar que, aunque este proyecto no esté enfocado a la implementación del Hardware IoT que realiza las mediciones y actuaciones, hay que tener en cuenta que el proyecto pueda ser implementado de manera completa en un entorno real. Es por ello por lo que en este apartado se van a tratar diferentes protocolos de comunicación enfocados a IoT y al bajo consumo que pueden ser de aplicación para este proyecto.

\paragraph{}
Asimismo, es interesante mencionar diferentes topologías de red que se pueden usar en IoT. Dependiendo de cada caso de uso específico en el que apliquen restricciones del terreno y el espacio, será necesario o más interesante emplear una u otra topología de red o modelo de comunicación.

\paragraph{}
A continuación se describen diferentes protocolos y/o tenologías de comunicación que son importantes dentro del paradigma de Internet de las Cosas:

\begin{itemize}
    \item LoRa: es una tecnologia inalámbrica con alta tolerancia a las interferencias, largo alcance y bajo consumo. Está gestionada actualmente por la LoRa Alliance. El protocolo que implementa dicha tecnología es LoRaWAN \cite{lora} \cite{loratech}.

    \item SigFox: es una solución para conectar dispositivos IoT. Es una solución propietaria y su precio se basa en el número de dispositivos conectados \cite{sigfox}.

    \item WiMax: IEEE 802.16, es un estándar de comunicación inalámbrica definido para redes de área metropolitana(MAN). \cite{wimax}

    \item Zigbee: IEEE 802.15.4 es un protocolo utilizado para redes de área local de bajo consumo orientado a dispositivos que funcionan mediante batería \cite{zigbee}.

    \item MQTT: es un protocolo de conectividad máquina a máquina diseñado para implementar un modelo de publicador/suscriptor. \cite{mqtt}

\end{itemize}

\end{document}
