\documentclass[../../memoria.tex]{subfiles}

\begin{document}

\paragraph{}
De nuevo es necesario mencionar que, aunque este proyecto no esté enfocado a la implementación del Hardware IoT que realice las mediciones y actuaciones, hay que tener en cuenta que el proyecto pueda ser implementado de manera completa en un entorno real. Es por ello por lo que en este apartado se van a tratar diferentes protocolos de comunicación enfocados a IoT y al bajo consumo que pueden ser de aplicación para este proyecto.

\paragraph{}
A continuación se describen diferentes protocolos y/o tenologías de comunicación que son importantes dentro del paradigma de Internet de las Cosas:

\begin{itemize}
    \item LoRa: es una tecnologia inalámbrica con alta tolerancia a las interferencias, largo alcance y bajo consumo. Está gestionada actualmente por la LoRa Alliance. El protocolo que implementa dicha tecnología es LoRaWAN \cite{lora} \cite{loratech}.

    \item SigFox: es una solución para conectar dispositivos IoT. Es una solución propietaria y su precio se basa en el número de dispositivos conectados \cite{sigfox}.

    \item WiMax: IEEE 802.16, es un estándar de comunicación inalámbrica definido para redes de área metropolitana(MAN). \cite{wimax}

    \item Zigbee: IEEE 802.15.4 es un protocolo utilizado para redes de área local de bajo consumo orientado a dispositivos que funcionan mediante batería \cite{zigbee}.

    \item MQTT: es un protocolo de conectividad máquina a máquina diseñado para implementar un modelo de publicador/suscriptor. \cite{mqtt}
\end{itemize}

\paragraph{}
Los aquí expuestos son solo un pequeño porcentaje de las alternativas que existen en la actualidad. En una posible línea de investigación futura, sería necesario realizar un estudio mucho más extenso sobre estos protocolos, así como diferentes pruebas de calidad, conectividad y consumo energético utilizando diversos protocolos y tecnologías. En este proyecto, únicamente se exponen algunos de ellos para tener una idea general de las alternativas, pero no se profundiza más porque ello podría componer un nuevo trabajo de la misma envergadura que este. Es una de las líneas de investigación futuras más disruptivas con la forma de realizar este proyecto, basado al 100\% en implementación de una solución para la parte Software.

\end{document}
