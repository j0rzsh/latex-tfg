\documentclass[../../memoria.tex]{subfiles}

\begin{document}

\paragraph{}
Por último, para finalizar este capítulo, se detallarán las características de la base software de este proyecto, la Infraestructura como Código o \textit{IaC}.

\paragraph{}
En los ejemplos descritos anteriormente se ha podido comprobar cómo uniendo la tecnología IoT junto con la agricultura se han podido desplegar y probar sistemas que conforman un ecosistema de agricultura de precisión. Sin embargo, todos estos ejemplos son puntuales, desarrollados para la prueba o diseñados \textit{ad-hoc} para un caso de uso concreto. La diferencia principal que se va a presentar en este trabajo es hacer uso de \textit{IaC} para desarrollar y desplegar la infraestructura que soporta un sistema de estas características.

\paragraph{}
La Infraestructura como Código es una tecnología que permite poder tratar una infraestructura de la misma manera que se puede tratar un programa software. Esto es, mediante código, versionable y reutilizable. Poder tratar la infraestructura como cualquier otro software aumentará la rapidez y eficiencia en los despliegues de la misma, reducirá la posibilidad de errores al automatizar tareas y ayudará a entender una infraestructura simplemente mirando el código que la define \cite{iac}\cite{iac2}\cite{iac3}.

\paragraph{}
Hacer uso de esta tecnología evita los problemas de falta de documentación sobre los recursos que existen desplegados en un determinado escenario. El código por sí mismo sirve como documentación. Permite además poder adaptar la infraestructura a las necesidades en cada momento de manera rápida y sencilla. Es además la mejor forma de gestionar los servicios que ofrece un proveedor de nube pública, pudiendo aprovechar al máximo las ventajas mencionadas anteriormente en este mismo capítulo. Ademaś, es una gran ayuda a la hora de no olvidar destruir recursos accidentalmente que generen gasto y repercuta negativamente en el coste.

\paragraph{}
Utilizar esta tecnología va a permitir replicar y adaptar el sistema a innumerables situaciones o escenarios que se planteen en diferentes campos de cultivo con diferentes necesidades. Es la utilización de \textit{IaC} lo que va a permitir que sin demasiado esfuerzo tanto de trabajo como económico se pueda desplegar un sistema que soporte un escenario real de agricultura de precisión.
\end{document}
