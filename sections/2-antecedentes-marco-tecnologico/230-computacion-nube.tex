\documentclass[../../memoria.tex]{subfiles}

\begin{document}

\paragraph{}
La Computación en la nube es un paradigma o modelo que se basa en ofrecer servicios de computación que pueden ser accedidos de manera ubicua y bajo demanda \cite{nist}. Estos servicios pueden componerse de diferentes elementos informáticos configurables, como por ejemplo redes, servidores, almacenamiento, servicios, etc. Estos recursos deben de poder crearse y destruirse de manera rápida con la menor gestión posible por parte del usuario final.

\paragraph{}
Dentro del modelo de Computación en la Nube, existen diferentes formas de ofrecer dichos servicios al usuario final, mediante diferentes modelos \cite{xaas}:

\begin{itemize}
    \item \uline{\textit{IaaS} (\textit{Infrastructure as a Service} o Infraestructura como Servicio)}: este modelo ofrece alternativas de infraestructura basadas en \textit{Cloud}. Dentro de este modelo se encuentran los servidores y las máquinas virtuales, servicios de almacenamiento de ficheros u objetos, elementos de red como firewalls o tablas de ruta, etc.

    \item \uline{\textit{PaaS} (\textit{Platform as a Service} o Plataforma como Servicio)}: mediante este modelo los proveedores \textit{Cloud} ofrecen una plataforma normalmente orientada al desarrollo, en la cual todas las dependencias o herramientas que pueda necesitar dicho desarrollo estén preinstaladas y preparadas para su uso, lo que permite ahorrar tiempo y dinero que se dedicaría a las instalaciones y configuraciones de dichas herramientas o dependencias. La característica principal de este modelo es que puede ser utilizado sin necesidad de poseer conocimiento técnico en la administración de sistemas.

    \item \uline{\textit{SaaS} (\textit{Software as a Service} o Software como Servicio)}: este modelo ofrece un servicio situado capas por encima de los anteriores mencionados. Se caracteriza por ofrecer un Software disponible para su uso de manera inmediata, que ofrece un servicio o responde a una necesidad específica. Dentro de este modelo se encuentran los proveedores de email, proveedores de almacenamiento en Nube, aplicaciones de gestión de usuarios, etc.
\end{itemize}

\paragraph{}
Comparar ventajas y desventajas de un modelo frente a otro no tendría sentido alguno, ya que cada modelo responde a una necesidad diferente, y es el caso de uso específico en cada momento lo que determinará qué modelo deberá ser usado teniendo en cuenta restricciones, costes, operativa, etc.

\paragraph{}
Por otro lado, es necesario mencionar que este tipo de computación trae consigo innumerables ventajas, sin embargo, hay que tener en cuenta los aspectos negativos de este modelo. Es imprescindible entender que la
Computación en la Nube no es un modelo que resulte ser el más eficiente en todos los aspectos para todos los casos de uso. Es importante también, adaptar la manera de realizar los diseños y las arquitecturas al modo de funcionar de la Computación en la Nube. Algunas características de la Computación en la Nube que pueden ser a su vez ventajas y desventajas, dependiendo de la forma en la que este modelo se utilice, son:

\begin{itemize}
    \item \uline{Costes}: siempre que los diseños y las implementaciones estén realizadas de manera eficiente, los costes de la infraestructura en la nube serán menores que los costes de los servidores físicos. La nube permite generar infraestructura con un coste muy bajo con respecto a lo que supondría comprar los recursos Hardware para realizar el mismo despliegue \textit{on-premises} (en las instalaciones físicas del usuario). Sin embargo, si, a modo de ejemplo, se desea migrar una infraestructura que está desplegada \textit{on-premises} directamente a los servidores de algún proveedor de Computación en la Nube, existe una gran probabilidad de que los costes sean incluso mayores a los que se tendrían en ese momento sin migrar los sistemas. Por tanto, es completamente imprescindible, si se quiere aprovechar al máximo las ventajas respecto a los costes que ofrece la Computación en la Nube, adaptar los sistemas que se quieran desplegar, diseñar y/o migrar a la forma de operar que tiene dicho paradigma. Es necesario estudiar primero los diferentes proveedores \textit{Cloud}, para después de haber elegido uno de ellos, estudiar los diferentes servicios que ofrece y sus precios asociados para así poder determinar cuál va a ser la mejor solución, tanto en rendimiento como en costes.

    \item \uline{Escalabilidad}: el hecho de no tener que adquirir físicamente infraestructura, convierte a la Computación en la Nube el mejor modelo para ofrecer una escalabilidad que se adapte a cualquier momento o entorno. Una gran ventaja de la Computación en la Nube es que es posible crear y destruir infraestructura, escalar (tanto hacia arriba como hacia abajo y tanto en vertical como en horizontal) dicha infraestructura libremente y automatizar este escalado para que se adapte en cada momento a la demanda que le exijan. En las infraestructuras \textit{on-premises}, es necesario conocer de antemano cual va a ser la demanda de tráfico hacia ese sistema, y cuáles pueden ser sus posibles picos, para no dejar de ofrecer servicio en ningún momento. Fallar en estas estimaciones puede suponer grandes pérdidas de dinero invertido en recursos que no soporten dichos picos de tráfico, y además, puede derivar en la necesidad de realizar una nueva compra de recursos. Sin embargo, en el modelo de Computación en la Nube, esto no ocurrirá debido a la naturaleza de los recursos que se utilizan. Estos recursos pueden crearse, destruirse y escalarse en todo momento para responder eficientemente a cualquier demanda en cualquier momento.

    \item \uline{Seguridad}: existen opiniones diversas sobre la seguridad en la nube. Este aspecto es decisivo para una gran cantidad de compañías que se niegan a adoptar el modelo de Computación en la Nube pensando que no es un entorno seguro. Sin embargo, la seguridad en la Nube necesita de una gestión similar a la seguridad de los sistemas \textit{on-premises}. Es necesario crear y proveer una seguridad en todo tipo de entornos y modelos. El hecho de utilizar la Nube no significa que un sistema esté más o menos seguro, ya que esto va a depender de la forma en la que dicha seguridad sea implementada y gestionada. Por otra parte, existe el inconveniente de que utilizar los servicios en Nube lleva consigo implícita una necesidad de confianza hacia el proveedor de dichos servicios. Es por esto que deben estudiarse a fondo los diferentes proveedores y elegir aquellos que generen una gran confianza tanto en seguridad como en fiabilidad del servicio. Otro aspecto a tener en cuenta sobre la seguridad es la privacidad de los datos y la información. Es necesario utilizar los servicios de aquellos proveedores que ofrezcan absoluta confidencialidad de los datos que se trasladen a su infraestructura. Este aspecto puede medirse investigando qué certificaciones de seguridad (ISO 27001, cumplimiento con HIPAA, etc.) Es necesario investigar cada proveedor \textit{Cloud} para conocer si sus servicios encajan o no con el caso de uso que se quiera aplicar.

    \item \uline{Vendor Lock-in}: quizás sea uno de los contras más importantes en el modelo de Computación en la Nube. Utilizar los servicios de un proveedor \textit{Cloud} exige sí o sí atarse a dicho proveedor. Esto puede generar problemas si en el futuro quiere migrarse a otro proveedor o quiere dejar de utilizarse el modelo de Computación en Nube. Por este motivo es importante realizar buenas estimaciones sobre costes, estudiar en profundidad qué ofrece cada proveedor y cómo puede adaptarse a las necesidades en cada momento. Si este aspecto es un gran inconveniente, se pueden adoptar estrategias como por ejemplo el uso de Nube Híbrida (parte del sistema en Nube Pública y parte en Nube Privada) o el uso de \textit{Multi-Cloud} (uso de diferentes proveedores \textit{Cloud}), teniendo en cuenta que estas estrategias aumentan exponencialmente los costes.

    \item \uline{Aspectos técnicos}: es necesario tener en cuenta también distintas características del uso de la Computación en la Nube como por ejemplo la necesidad de contar con una conexión a Internet en todo momento para poder acceder a dichos servicios. En la actualidad, esto no debería suponer un problema para prácticamente todo tipo de escenario o caso de uso, pero es un aspecto que también debe tenerse en cuenta. Además, es imprescindible tener en cuenta que, al no existir acceso físico a los servidores o elementos del sistema, no es posible pararlos cortando su fuente de energía, o no pueden realizarse reparaciones de forma física sobre ellos.
\end{itemize}

\paragraph{}
Para el caso de uso que se va a estudiar en este trabajo, únicamente se planteará la utilización de un proveedor de Nube Pública (es decir, los servidores o servicios utilizados serán compartidos con el resto de los clientes del proveedor, aislados únicamente de manera lógica unos de otros y no de manera física).

\paragraph{}
Otro aspecto interesante de la Computación en la Nube es que (dependiendo del proveedor) la infraestructura puede ser gestionada utilizando el modelo de Infraestructura como Código, del cual se hablará más adelante en este trabajo, en el capítulo de Descripción de la Solución Propuesta. En ese mismo capítulo se tratarán más a fondo diferentes temas aquí expuestos, adaptados ya al proveedor elegido para el despliegue del sistema.

\end{document}
