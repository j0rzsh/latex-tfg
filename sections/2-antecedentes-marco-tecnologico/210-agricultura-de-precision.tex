\documentclass[../../memoria.tex]{subfiles}

\begin{document}
\paragraph{}
La agricultura de precisión es el proceso mediante el cual se monitorizan las posibles variables que puedan existir en el proceso de cultivo de un campo agrícola (como por ejemplo la humedad del aire, la temperatura del suelo, etc.) para poder llevar un control sobre estas variables y también poder decidir una futura actuación en función de dichas métricas.

\paragraph{}
La agricultura de precisión se puede dividir en cuatro fases distintas que estarán activas durante todo el proceso de cultivo y una fase más que se realizará de manera periódica (basado en \cite{qampo}):

\begin{enumerate}
  \item Recogida de datos: en esta fase los dispositivos elegidos para la monitorización de las diferentes métricas, variables o elementos del campo de cultivo se encargan de recoger datos. Es la primera fase y de la que nace todo el proceso de la agricultura de precisión. Es la fase esencial ya que la precisión de la recogida de datos es la que va a marcar la calidad final del sistema. Si se tiene una precisión deasiado baja, no va a poder solucionarse con el resto del sistema a implementar, ya que hará de cuello de botella para la calidad final de la solución. Es por esto que es imprescindible contar con un equipo Hardware suficientemente preciso para más adelante poder realizar el resto de las fases de una manera eficiente y que realmente aporte un valor a la agricultura actual.

  \item Almacenamiento y pre procesado de los datos: una vez recogidos y preprocesados (si fuese necesario) los datos que miden los dispositivos que actúan como sensores, es necesario disponer de un sistema de almacenamiento dedicado a guardar todos estos datos. Este punto es crucial para el futuro desarrollo del sistema, ya que elegir un método de almacenamiento poco eficiente, caro o no adecuado para el caso de uso, va a condicionar las fases de análisis. Es necesario ajustar este almacenamiento de los datos teniendo en cuenta el tipo de datos a recoger, la forma de indexado que se va a utilizar y el tamaño de cada unidad de datos que se defina. En el capítulo dedicado a la solución propuesta, se argumentará el almacenamiento escogido teniendo en cuenta, sobre todo, el posible escalado que pueda sufrir.

  \item  Análisis de los datos: la fase de análisis de datos es la que va a decidir las actuaciones y la que va a permitir aportar el valor añadido a la agricultura. En esta fase de análisis es crucial entender las necesidades que pueda tener el agricultor y el trabajo que realice. Es necesario comprender qué aspectos son esenciales para una agricultura eficiente, en qué es necesario enfocar los esfuerzos del sistema y cómo se detectan posibles problemas mediante estos análisis de la información obtenida de las medidas de los sensores.

  \item  Toma de decisiones. Actuación: esta fase toma como base las salidas o conclusiones de la anterior para automatizar procesos llevados a cabo en la agricultura. En esta fase la lógica implementada debe ser mínima, ya que el grueso del procesado de datos se encuentra en la anterior. Es importante simplificar al máximo la lógica en esta fase ya que en los actuadores Hardware es el lugar en el que más dificultades se van a encontrar las automatizaciones de procesos. Al igual que en la anterior fase, es necesario conocer las necesidades del agricultor, y los procesos que lleva a cabo dependiendo de su experiencia y observaciones. El objetivo es conseguir la máxima automatización posible, por lo tanto, es imprescindible entender los procesos de la agricultura y plantear soluciones que mejoren dichos procesos.

  \item  Evaluación del rendimiento: al contrario que las cuatro primeras fases que se encuentran activas el 100\% del tiempo, esta fase se debe llevar de manera periódica para estudiar cómo de buena es la solución, qué problemas mejora y cuáles no, y qué remedios hay que estudiar en cada momento para que realmente el sistema implementado no sólo automatice las labores del agricultor, sino que maximice los resultados de la agricultura. En esta fase es importante contar con la ayuda y opinión del agricultor, que aportará una visión diferente a la visión técnica del sistema y adaptada a la agricultura en sí misma.
\end{enumerate}

\paragraph{}
Para este trabajo, se va a hacer hincapié en las tres primeras fases, teniendo en cuenta que para la primera fase únicamente se realizarán simulaciones de los dispositivos por motivos de presupuesto y logística. Es objetivo de este trabajo el presentar una solución que dé respuesta a la problemática que presentan las fases dos y tres, con especial foco en la segunda fase. Una vez presentada e implementada la solución, se elaborará una lista de pasos a seguir con ideas de implementación para las siguientes fases.

\paragraph{}
No es objetivo de este trabajo la problemática que presenta la gestión de los dispositivos Hardware: energía limitada, ancho de banda limitado, adaptación a las condiciones atmosféricas y meteorológicas, tipos de sensores, etc., pero si se van a exponer ejemplos de sensores y tecnologías que podrían encajar en la primera fase de la agricultura de precisión aquí mencionada.
\end{document}
