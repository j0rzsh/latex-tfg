\documentclass[../../memoria.tex]{subfiles}

\begin{document}

\paragraph{}
Para la realización de este proyecto, como ya se ha mencionado anteriormente, el desarrollo se va a enfocar en el diseño de la infraestructura \textit{Cloud} que soporte el sistema. Sin embargo, es necesario conocer que existen soluciones \textit{hardware} actualmente en el mercado que den respuesta a la parte física del sistema, en la que se desplegarían los diferentes sensores y/o actuadores IoT.

\paragraph{}
Existen innumerables fabricantes y dispositivos que podrían usarse para un sistema como el que se presenta en este proyecto, sin embargo, se han recopilado aquí dos tipos de dispositivos de dos fabricantes diferentes, que pueden dar respuesta al sistema de manera distinta:

\begin{itemize}

    \item Raspberry Pi Zero W y similares \cite{raspberry}: Raspberry Pi es un ordenador en miniatura, montado sobre una placa base única que posee una serie de pines de entrada y salida, aparte de conexiones USB y HDMI. Posee una gran comunidad como soporte que ayuda a los desarrolladores a compartir sus soluciones y resolver los errores que encuentren en el desarrollo. La Raspberry Pi es capaz incluso de ejecutar un sistema operativo (Raspbian) con un entorno gráfico. Existen numerosas alternativas a Raspberry con la misma meta: una placa de propósito general.

    \item Dispositivos de Libelium \cite{libelium}: Libelium es un fabricante especializado en \textit{IoT}. Sus dispositivos están mucho más orientados a propósitos bien definidos. Es por ello que, para un sistema en fase de producción, dispositivos como estos son la mejor alternativa. Están diseñados y preparados para una función específica.
\end{itemize}

\paragraph{}
En resumen, hablando de \textit{hardware} para \textit{IoT}, existen muchas soluciones de diferentes fabricantes los cuales tienen diferentes propósitos a la hora de fabricar sus dispositivos. En un proceso de desarrollo de un proyecto como este, es importante conocer ambos tipos de dispositivos, ya que ambos serán útiles en distintas fases del desarrollo e investigación. Por ello, una recomendación personal es conseguir un dispositivo de propósito general, asequible, para poder realizar diferentes pruebas de concepto y comprobar que la infraestructura (en este caso en nube pública) que soporta el sistema completo está diseñada correctamente y se adecúa a los requerimientos. Una vez bien definido el sistema e implementado la parte independiente del Hardware, sería interesante tener a disposición dispositivos reales de producción, para poder realizar los diferentes desarrollos a nivel Software que estos requieran, así como pruebas de conectividad y gasto energético.

\paragraph{}
En la última parte de este proyecto, en el capítulo de trabajos futuros y próximos pasos, se detallará un ejemplo de posible desarrollo futuro que involucre un dispositivo \textit{hardware} real y no simulado. Para el desarrollo de este proyecto se pretende únicamente simular estos dispositivos con un pequeño desarrollo \textit{software} que ayude a realizar los test pertinentes de funcionamiento de la infraestructura.

\end{document}
