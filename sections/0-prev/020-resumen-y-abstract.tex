% !TEX root = memoria.tex
\documentclass[../../memoria.tex]{subfiles}

\begin{document}

\paragraph{}
Actualmente la tecnología está presente en prácticamente cualquier escenario, actividad, negocio, situación, etc. Se han modernizado la mayor parte de los trabajos y actividades productivas de la sociedad. En esta transformación hacia el mundo digital se ha comenzado a incluir a una de las actividades más antiguas del ser humano: la agricultura.

\paragraph{}
La inclusión de la tecnología en la agricultura facilita una gran cantidad de tareas llevadas a cabo en las tierras de cultivo, sin embargo, aún es un campo con un gran camino a recorrer. El auge de la computación en la nube y los dispositivos IoT ha permitido nuevas formas de aplicar esta tecnología a la agricultura. Es así como nace la agricultura de precisión.

\paragraph{}
Gracias a la agricultura de precisión, es posible controlar una plantación y actuar sobre ella de la manera más eficiente posible, lo que lleva a un mejor desarrollo de dicha actividad. En este proyecto se propone una forma de trabajo y unas tecnologías y herramientas para ello que incluyen el uso de la nube pública, uso de dispositivos IoT y uso de Infraestructura como Código, para poder desarrollar un sistema completo que dé solución a la necesidad de monitorización de una tierra de cultivo. Se ha desarrollado pensando tanto en su escalabilidad, es su integración con otras tecnologías y en el precio total del sistema desplegado.

\paragraph{}
Este sistema servirá de base para establecer una monitorización completa sobre variables como la presión atmosférica, la temperatura y humedad del aire, la humedad del terreno, etc. que mediante su posterior análisis sea posible establcer actuaciones automatizadas sobre los campos de cultivo que mejoren de manera abismal la eficiencia de la agricultura actual.
\end{document}
