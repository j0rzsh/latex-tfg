\documentclass[../../memoria.tex]{subfiles}

\begin{document}

\paragraph{}
Con el fin de automatizar todo lo posible el despliegue de la solución, se ha decidido crear varios scripts que sirvan de ayuda para ello. Estos scripts son:

\begin{enumerate}
    \item launch\_tfg.sh: este script está escrito en Bash y se encarga de automatizar los comandos necesarios para el despliegue. Esto es:
          \begin{itemize}
              \item Crea un \textit{workspace} de Terraform y ejecuta el comando \textit{terraform apply -auto-approve} que omite la confirmación del usuario antes de aplicar los cambios.
              \item Ejecuta el script que se explica en el siguiente punto (save\_outputs\_terraform.py) para guardar las variables de salida de Terraform y poder usarlas tanto en el simulador como en el resto de herramientas que sean necesarias.
              \item Recoge la IP de la instancia EC2 que se va a utilizar para ejecutar el simulador.
              \item Copia mediante scp el código del simulador en la instancia. Ejecuta mediante ssh el comando \textit{docker build} y el comando \textit{docker run} para ejecutar el simulador.
          \end{itemize}

    \item save\_outputs\_terraform.py: este script recoge la salida de Terraform una vez desplegada la infraestructura. Guarda en una carpeta con nombre config los siguientes datos:
          \begin{itemize}
              \item Certificados de los dispositivos IoT. Estos certificados se almacenan en el \textit{tfstate} mencionado anteriormente. Dicho \textit{tfstate} se encuentra cifrado dentro del Bucket S3 por lo que contiene la seguridad necesaria para el almacenaje de dichos certificados. Tal y como se describe en \cite{terraformsensitive}, se trata el \textit{tfstate} como datos sensibles y por tanto se controla su acceso mediante políticas de IAM y cifrado desde el lado del servidor.

              \item Punto de acceso del dominio de Elasticsearch: se guarda para poder acceder a la base de datos.

              \item Punto de acceso IoT de AWS: se guarda para poder pasárselo como parámetro al simulador.
          \end{itemize}

    \item destroy\_tfg.sh: este script simplemente ejecuta el comando \textit{terraform destroy -auto-approve} para destruir toda la infraestructura creada sin tener que pedir confirmación al usuario.
\end{enumerate}

\end{document}
